\documentclass[12pt]{article}
\usepackage[utf8]{inputenc}
\usepackage{upquote}
\usepackage[margin=1in]{geometry} 
\usepackage{amsmath,amsthm,amssymb}
\usepackage{graphicx}
\usepackage{listings}
\newenvironment{statement}[2][Statement]{\begin{trivlist}
\item[\hskip \labelsep {\bfseries #1}\hskip \labelsep {\bfseries #2.}]}{\end{trivlist}}
\usepackage{booktabs}
\usepackage{multirow}
\usepackage{subfigure}


% Listings package for code rendering (No external dependencies)
\usepackage{listings}  
\usepackage{xcolor}   % Color support
\usepackage{tcolorbox} % Box for better appearance

% Define custom colors for code highlighting
\definecolor{codegreen}{rgb}{0,0.6,0}
\definecolor{codegray}{rgb}{0.5,0.5,0.5}
\definecolor{codepurple}{rgb}{0.58,0,0.82}
\definecolor{backcolour}{rgb}{0.95,0.95,0.92}


\lstset{frame=tb,
    language=Python,
    backgroundcolor=\color{backcolour},   
    commentstyle=\color{codegreen},
    keywordstyle=\color{magenta},
    numberstyle=\tiny\color{codegray},
    stringstyle=\color{codepurple},
    basicstyle=\ttfamily\footnotesize,
    breakatwhitespace=false,         
    breaklines=true,                 
    keepspaces=true,                 
    numbers=left,       
    numbersep=5pt,                  
    showspaces=false,                
    showstringspaces=false,
    showtabs=false,                  
    tabsize=2,
}


\title{Handin 2}

\begin{document}
\maketitle

\section{Introduction}

We investigate the implementation and benchmarking of the Steepest Descent and Newton's Method, both using backtracking line search. We aim to analyze their convergence behavior and compare their efficiency.  We also solve theoretical problems that provide insights into the algorithmic choices and expected performance. 

In our benchmark protocol, we focus on the five functions provided in the case study. 

\section{Experiments}

\subsection{Experimental Setup: Stopping Criteria}
% Discuss what stopping criteria you use and why? How do you pick the values for the stopping thresholds?
We apply two kinds of stopping criteria, including the gradient norm threshold and the maximum iteration threshold. We use the gradient norm threshold to ensure the algorithm stops when the gradient is sufficiently small. We use the max iteration to limit the algorithm to run infinitely when there is ill conditions, in case the algorithm would run definitely.

\begin{lstlisting}
# Max Iteration Number
for _ in range(max_iter): 
    grad = grad_f(x)
    # Gradient Norm Threshold 
    if np.linalg.norm(grad) < tol:
        break
\end{lstlisting}

\textbf{Stopping Threshold Value Choice: }We choose the tolerance gradient value=1e-6 because it is small enough and will not cost too much iterations. We choose max iterations=1000, because the gradient norm threshold will reach firstly in our experiment, so the max iterations will not effect the gradient norm threshold. 

\subsection{Experimental Setup: Parameter Selection }
% How do you select parameter values for the backtracking line search? Is your result sensitive towards these settings?
Here we discuss our parameter selections and their logic in Backtracking Line Search. As the Armijo issues:

\begin{equation}
\begin{aligned}
if \, & f(x_k + \alpha d_k) \leq f(x_k) + c_1 \alpha \nabla f(x_k)^T d_k:\\
& \alpha = \rho * \alpha
\end{aligned}
\end{equation}

There are 3 selectable variables in Backtracking Line Search, as listed below. The initial searching point $x_0$ also matters, but we will not discuss much since we have done this in hand-in1. So we directly use the selection of $x_0$ from hand-in1's conclusion.

\begin{enumerate}
  \item The initial step size $\alpha_0$: we unify $\alpha_0$ to 1.0.
  \item The sufficient decrease parameter $c_1$: we unify $c_1$ to $1e-4$.
  \item The step shrinking factor$\rho$: we select different $\rho$, from ${0.1,0.3,0.5,0.9}$.
\end{enumerate}

\subsubsection{Selection of $\alpha_0$}
We consider the $\alpha_0$ choice based on the different step scales in Newton's and Steepest Decent algorithm.

For Newton's method, since Hessian is considered in the step direction, it can choose a naturally satisfying-Armijo-condition step when the Hessian is well-conditioned. The function decrease is typically large enough that $\alpha_0=1.0$ satisfies the Armijo condition without backtracking. When the Hessian is ill-conditioned, backtracking is required, and $\alpha$ wii reduce.

For Steepest Decent, the search direction does not consider the second-order information, so the step size is often poorly scaled. A full step like $\alpha_0=1.0$ may fail to satisfy the Armijo condition. But it just lead to a few more back tracking, so for convenience we just unify the $\alpha_0=1.0$ in our benchmark protocol.

\subsubsection{Selection of $c_1$}

The parameter $c_1$ controls how much decrease is required in $f(x)$ per step. A smaller $c_1$  would require a more significant decrease, leading to excessive shrinking of $\alpha$ and slower convergence. A larger $c_1$ would accept even small improvements and might allow larger steps, but this could lead to instability.

For Newton's Method, the step is well-scaled when the Hessian is well-conditioned, and the Armijo condition is often easily satisfied. So Newton's Method is less sensitive to $c_1$.

For Steepest Descent, it is sensitive to $c_1$'s choice. Because its gradient steps are poorly scaled, so $c_1$ cannot be too large. $1e-4$ is a common choice among the standard library, for example, in scipy.optimization.line\_search. 

Since the two algorithms could reach a common need for $c_1$, we just unify the $c_1$ in the benchmark protocol to $1e-4$.

\subsubsection{Selection of $\rho$}

For Newton's Method, backtracking is rarely needed when the Hessian is well-conditioned, and when backtracking is required, a larger $\rho$ is preferred because Newton's step is often in the right direction, and only small adjustment is needed.

For Steepest Descent, since it often produces steps that violate the Armijo condition, backtracking is frequently needed. A larger $\rho$ makes the step size shrink slowly, leading to more backtracking iterations. So a smaller $\rho$ is preferred, so that the $\alpha$ could shrink quickly.

Since the two algorithms have different preference to $rho$, and both of them are somehow sensitive to the $\rho$, our benchmark protocols covers different $\rho$. 

\section{Results and Discussion}

Our benchmark protocol cover the following two features: 

\begin{enumerate}
  \item Different target functions, particularly the functions provided in the case\_study.py;
  \item Different backtracking step lengths($\rho$).
\end{enumerate}

\subsection{Correctness Verification}
% Evaluate whether your results are correct. Describe the steps you took to ensure correctness of your algorithms and show data that supports your claims.

\subsection{Step Length Observations}
% For both algorithms, describe your observations of the chosen step lengths. Are there important differences between the functions and algorithms?


Newton is not sensitive to the $\rho$ change.  % TODO reasoning

\subsection{Convergence Plots and Analysis}
% Show convergence plots and based on the empirical results, state whether the algorithms likely have Q-linear, Q-superlinear or maybe even Q-quadratic convergence on the different functions.

The convergence plots is shown in Figure~\ref{fig:convgraph}. 

For f1, Newton's method vertically reaches optimum in 1 iteration, and it shows a Q-quadratic convergence. Steepest Descent  shows a slower convergence and it is a Q-Linear convergence.

For f2, under Newton’s Method it is Q-quadratic convergence near the optimum, because it rapidly drop to zero after a few iterations. Under Steepest Descent, there is a slow, oscillatory decay, so it is Q-linear convergence.



\begin{figure}[ht]
\centering
\subfigure[]{
    \includegraphics[width=0.25\columnwidth, keepaspectratio]{code_week2/f1}
\label{fig:f1}
}
\subfigure[]{
    \includegraphics[width=0.25\columnwidth, keepaspectratio]{code_week2/f2}
    \label{fig:f2}
}
\subfigure[]{
    \includegraphics[width=0.25\columnwidth, keepaspectratio]{code_week2/f3}
    \label{fig:f3}
}
\subfigure[]{
    \includegraphics[width=0.25\columnwidth, keepaspectratio]{code_week2/f4}
    \label{fig:f4}
}
\subfigure[]{
    \includegraphics[width=0.25\columnwidth, keepaspectratio]{code_week2/f5}
    \label{fig:f5}
}
\caption[]{Convergence graph(log scale) of the three optimize algorithms on the five functions, analyzing iterations}
\label{fig:convgraph}
\end{figure}


\section{Theory}

\subsection{Question 1: Sum of Different Powers Function}
% TODO jiayi

\subsection{Question 2: Sufficient Decrease Condition in 1D}
% TODO wendong

\section{Conclusion}


\end{document}
