\documentclass[12pt]{article}
\usepackage[utf8]{inputenc}
\usepackage{upquote}
\usepackage[margin=1in]{geometry} 
\usepackage{amsmath,amsthm,amssymb}
\usepackage{graphicx}
\usepackage{listings}
\newenvironment{statement}[2][Statement]{\begin{trivlist}
\item[\hskip \labelsep {\bfseries #1}\hskip \labelsep {\bfseries #2.}]}{\end{trivlist}}
\usepackage{xcolor}


% Listings package for code rendering (No external dependencies)
\usepackage{listings}  
\usepackage{xcolor}   % Color support
\usepackage{tcolorbox} % Box for better appearance

% Define custom colors for code highlighting
\definecolor{codegreen}{rgb}{0,0.6,0}
\definecolor{codegray}{rgb}{0.5,0.5,0.5}
\definecolor{codepurple}{rgb}{0.58,0,0.82}
\definecolor{backcolour}{rgb}{0.95,0.95,0.92}


\lstset{frame=tb,
    language=Python,
    backgroundcolor=\color{backcolour},   
    commentstyle=\color{codegreen},
    keywordstyle=\color{magenta},
    numberstyle=\tiny\color{codegray},
    stringstyle=\color{codepurple},
    basicstyle=\ttfamily\footnotesize,
    breakatwhitespace=false,         
    breaklines=true,                 
    keepspaces=true,                 
    numbers=left,       
    numbersep=5pt,                  
    showspaces=false,                
    showstringspaces=false,
    showtabs=false,                  
    tabsize=2,
}


\title{Handin 2}

\begin{document}
\maketitle

\section{Introduction}

We investigate the implementation and benchmarking of the Steepest Descent and Newton's Method, both using backtracking line search. We aim to analyze their convergence behavior and compare their efficiency.  We also solve theoretical problems that provide insights into the algorithmic choices and expected performance.

\section{Experiments}

\subsection{Experimental Setup: Stopping Criteria}
% Discuss what stopping criteria you use and why? How do you pick the values for the stopping thresholds?
We apply two kinds of stopping criteria, including the gradient norm threshold and the maximum iteration threshold. We use the gradient norm threshold to ensure the algorithm stops when the gradient is sufficiently small. We use the max iteration to limit the algorithm to run infinitely when there is ill conditions, in case the algorithm would run definitely.

\begin{lstlisting}
# Max Iteration Number
for _ in range(max_iter): 
    grad = grad_f(x)
    # Gradient Norm Threshold 
    if np.linalg.norm(grad) < tol:
        break
\end{lstlisting}

\textbf{Value Choice: }We choose the tolerance gradient value=1e-6 because it is small enough and will not cost too much iterations. We choose max iterations=1000, because the gradient norm threshold will reach firstly in our experiment, so the max iterations will not effect the gradient norm threshold. 

\subsection{Parameter Selection in Backtracking Line Search }
% How do you select parameter values for the backtracking line search? Is your result sensitive towards these settings?

As the Armijo issues: 

\begin{equation}
f(x_k + \alpha d_k) \leq f(x_k) + c_1 \alpha \nabla f(x_k)^T d_k
\end{equation}

We set initial step(alpha) as 1.0 in Newton's, because the quadratic model assumption often suggests a full step is reasonable. We set it 0.1 in Steepest Descent, using a smaller initial value to avoid overshooting.

We set the step shrink ratio ($\rho$) is 0.5. This param controls how aggressively the step size is reduced. The smaller it is, the faster the algorithm will shrink, while wasting evaluations.  Take rho=0.5 and rho=0.8 as an example, the step to reach the same step length $\alpha$ is:
\begin{equation}
\begin{aligned}
k_{0.5}&=log_{\rho}(\alpha / \alpha_0) = \frac{ln((\alpha / \alpha_0)}{ln(0.5)} \\
k_{0.8}&=\frac{ln((\alpha / \alpha_0)}{ln(0.8)} 
\end{aligned}
\end{equation}
Since $|ln(0.5)| > |ln(0.8)|$, so the smaller the $\rho$ is, the less steps the algorithm will take. 

We also have the Wolfe condition param, Sufficient Decrease Parameter(c1), set default by 1e-4. This parameter ensures the step produces a meaningful function decrease, and the smaller the value is, the more permissive the condition is.


\section{Results and Discussion}

\subsection{Correctness Verification}
% Evaluate whether your results are correct. Describe the steps you took to ensure correctness of your algorithms and show data that supports your claims.

\subsection{Step Length Observations}
% For both algorithms, describe your observations of the chosen steplengths. Are there important differences between the functions and algorithms?

\subsection{Convergence Plots and Analysis}
% Show convergence plots and based on the empirical results, state whether the algorithms likely have Q-linear, Q-superlinear or maybe even Q-quadratic convergence on the different functions.


\section{Theory}

\subsection{Question 1: Sum of Different Powers Function}
% Derivation of Newton step and convergence properties

\subsection{Question 2: Sufficient Decrease Condition in 1D}
% Proof for sufficient decrease condition in polynomial functions


\section{Conclusion}


\end{document}
